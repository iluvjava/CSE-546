\documentclass[]{article}
\usepackage{amsmath}\usepackage{amsfonts}
\usepackage[margin=1in,footskip=0.25in]{geometry}
\usepackage{mathtools}
\usepackage{hyperref}
\hypersetup{
    colorlinks=true,
    linkcolor=blue,
    filecolor=magenta,
    urlcolor=cyan,
}
\usepackage[final]{graphicx}
\usepackage{listings}
\usepackage{courier}
\lstset{basicstyle=\footnotesize\ttfamily,breaklines=true}

% \usepackage{wrapfig}
\graphicspath{{.}}

\begin{document}
\begin{center}
    Name: Honda Li \quad Class: CSE 546 SPRING 2021\quad HW3B 
\end{center}

\section*{B.1: Intro to Sample Complexity}
    \subsection*{B.1.a}
        Let's exam the statement: 
        \begin{align*}\tag{B.1.a}\label{eqn:B.1.a}
            \mathbb{P}\left[
                \hat{R}_n(f) = 0
            \right]
            &= 
            \mathbb{P}\left[
                \frac{1}{n}\sum_{i = 1}^{n}
                    \mathbf{1}\{
                        f(x_i) \ne y_i
                    \} = 0
            \right]
            \\
            &= 
            \prod_{i = 1}^{n}
                1 - \mathbb{P}\left[
                    f(x_i) \ne y_i
                \right]
        \end{align*}
    Let's use the statement in the hypothesis. The statement was $R(f) > \epsilon$, which describes the event that $\mathbb{E}\left[\mathbf{1}\{f(x)\ne Y\}\right]$. so then $1 - R(f)$ describes exepected value of the event that: $1 - \mathbb{P}\left(f(x_i) \ne y\right)$. And notice that $1 - R(f)< 1 - \epsilon < \exp(epsilon)$, so then we can simplify the above expression into: 
    \begin{align*}\tag{B.1.a.1}\label{eqn:B.1.a.1}
        \prod_{i = 1}^{n}
        1 - \mathbb{P}\left[
            f(x_i) \ne y_i
        \right] = (1 - R(f)) \le (\exp(\epsilon))^n = \exp(n\epsilon)
        \\
        \implies 
        \mathbb{P}\left[
            \hat{R}_n(f) = 0
        \right] \le \exp(n\epsilon)
    \end{align*}

    \subsection*{B.1.b}    
        The results from the previous involves the hypothesis that $R(f)\ge \epsilon$, the theoretical risk of the model is larger than $\epsilon$, therefore, under a larger scope the more appropriate inequality to make should be: 
        \begin{align*}\tag{B.1.b.1}\label{eqn:B.1.b.1}
            \mathbb{P}\left[
                \hat{R}_n(f) = 0\wedge R(f) \ge \epsilon
            \right] \le \exp(-n\epsilon)
        \end{align*}
        For the proof, let's start with the following statement: 
        \begin{align*}\tag{B.1.b.2}\label{eqn:B.1.b.2}
            \mathbb{P}\left[
                f(f) > \epsilon \wedge \hat{R}_n(f) = 0
            \right] &= 
            \mathbb{P}\left[\left.
                \hat{R}_n(f) = 0 \right|R(f) > \epsilon
            \right]\mathbb{P}\left[
                R(f)> \epsilon
            \right]
            \\
            \implies \mathbb{P}\left[
                f(f) > \epsilon \wedge \hat{R}_n(f) = 0
            \right]&\le \mathbb{P}\left[\left.
                \hat{R}_n(f) = 0 \right|R(f) > \epsilon
            \right] \le \exp(-n\epsilon)
        \end{align*}
        Now, the statement $\exists f\in \mathcal{F}: R(f)> \epsilon \wedge \hat{R}_n(f) = 0$ implies that occurence of at least one events, occurence of many events is the includes the case of at least one event, therefore we can say that: 
        \begin{align*}\tag{B.1.b.3}\label{eqn:B.1.b.3}
            \mathbb{P}\left[
                \exists f \in \mathcal{F}: R(f) > \epsilon \wedge \hat{R}_n(f) = 0 
            \right] 
            &\le 
            \mathbb{P}\left[
                \bigcup_{f\in \mathcal{F}} \left\lbrace 
                R(f) > \epsilon \wedge \hat{R}_n(f) = 0
                \right\rbrace
            \right]
            \\
            &\underset{\text{Union Bound}}{\le}
            \sum_{f\in \mathcal{F}}^{}
            \mathbb{P}\left[
                R(f) > \epsilon \wedge \hat{R}_n(f) = 0
            \right]
            \\
            &\le\left|
                \mathcal{F}
            \right|\exp(-n\epsilon)
        \end{align*}
    \section*{B.1.c}
        

\end{document}
