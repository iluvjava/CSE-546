\documentclass[]{article}
\usepackage{amsmath}\usepackage{amsfonts}
\usepackage[margin=1in,footskip=0.25in]{geometry}
\usepackage{mathtools}
\usepackage{hyperref}
\hypersetup{
    colorlinks=true,
    linkcolor=blue,
    filecolor=magenta,
    urlcolor=cyan,
}
\usepackage[final]{graphicx}
\usepackage{listings}
\usepackage{courier}
\lstset{basicstyle=\footnotesize\ttfamily,breaklines=true}

% \usepackage{wrapfig}
\graphicspath{{.}}

\begin{document}
\begin{center}
    Name: Honda Li \quad Class: CSE 546 SPRING 2021\quad HW3B 
\end{center}

\section*{B.1: Intro to Sample Complexity}
    \subsection*{B.1.a}
        Let's exam the statement: 
        \begin{align*}\tag{B.1.a}\label{eqn:B.1.a}
            \mathbb{P}\left[
                \hat{R}_n(f) = 0
            \right]
            &= 
            \mathbb{P}\left[
                \frac{1}{n}\sum_{i = 1}^{n}
                    \mathbf{1}\{
                        f(x_i) \ne y_i
                    \} = 0
            \right]
            \\
            &= 
            \prod_{i = 1}^{n}
                1 - \mathbb{P}\left[
                    f(x_i) \ne y_i
                \right]
        \end{align*}
    Let's use the statement in the hypothesis. The statement was $R(f) > \epsilon$, which describes the event that $\mathbb{E}\left[\mathbf{1}\{f(x)\ne Y\}\right]$. so then $1 - R(f)$ describes exepected value of the event that: $1 - \mathbb{P}\left(f(x_i) \ne y\right)$. And notice that $1 - R(f)< 1 - \epsilon < \exp(epsilon)$, so then we can simplify the above expression into: 
    \begin{align*}\tag{B.1.a.1}\label{eqn:B.1.a.1}
        \prod_{i = 1}^{n}
        1 - \mathbb{P}\left[
            f(x_i) \ne y_i
        \right] = (1 - R(f)) \le (\exp(\epsilon))^n = \exp(n\epsilon)
        \\
        \implies 
        \mathbb{P}\left[
            \hat{R}_n(f) = 0
        \right] \le \exp(n\epsilon)
    \end{align*}

    \subsection*{B.1.b}    
        
    

\end{document}
