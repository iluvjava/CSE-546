\documentclass[]{article}
\usepackage{amsmath}\usepackage{amsfonts}
\usepackage[margin=1in,footskip=0.25in]{geometry}
\usepackage{mathtools}
\usepackage{hyperref}
\hypersetup{
    colorlinks=true,
    linkcolor=blue,
    filecolor=magenta,
    urlcolor=cyan,
}
\usepackage[final]{graphicx}
\usepackage{listings}
\usepackage{courier}
\lstset{basicstyle=\footnotesize\ttfamily,breaklines=true}

% \usepackage{wrapfig}
\graphicspath{{.}}

\begin{document}
\begin{center}
    Name: Hongda Li \quad Class: CSE 546 \quad HW2B
\end{center}
\section*{B.1}
    \textbf{Objective}: Given the definition for the L2, L1 and the Infinity norm of real vector, show that $\Vert x\Vert_\infty \le \Vert x\Vert_2 \le \Vert x\Vert_1$. 
    \\
    First we are going to show that $\Vert x\Vert_2^2 \le \Vert x\Vert_1^2$, starting from the definition of the norms we have: 
    \begin{align*}\tag{B.1.1}\label{eqn:B.1.1}
        \Vert x\Vert_1^2 &= 
            \left(
                \sum_{i = 1}^{n} |x_i|
            \right)^2
        \\ 
        &= \sum_{i = 1}^{n}
            \left(
                |x_i|\sum_{j = 1}^{n}
                    |x_j|
            \right)
        \\
        &= 
        \sum_{i = 1}^{n}
            \left(
                |x_i|^2 + 
                |x_i|\sum_{j = 1, j \ne i}^{n}
                    |x_j|
            \right)
        \\
        &= 
        \sum_{i = 1}^{n} |x_i|^2 + \sum_{i = 1}^{n}|x_i|\left(
            \sum_{ j= 1,j \ne i }^{n}
                |x_j|
        \right)
        \\
        &= 
        \Vert x\Vert_2^2 + \underbrace{\sum_{i = 2}^{n}\sum_{j = 1}^{i - 1}2|x_i||x_j|}_{\ge 0}
        \\
        &\implies \Vert x\Vert_2^2 \le \Vert x\Vert_1^2
    \end{align*}
    And now we are going to shoe that $\Vert x\Vert_\infty^2 \le \Vert x\Vert_2^2$. By the definition of the infinity norm, we know that therde exists $1\le m \le n$ such that $x_m = \Vert x\Vert_\infty = \max_{1\le i\le n}(x_i)$. Then it can be said that: 
    \begin{align*}\tag{B.1.2}\label{eqn:B.1.2}
        x_m^2 \le& x_m^2 + \underbrace{\sum_{i = 1, i\ne m}^{n} x_i^2}_{\ge 0}
        \\
        x_m^2 =& \Vert x\Vert_\infty \le \sum_{i = 1}^{n}x_i^2 = \Vert x\Vert_2^2
    \end{align*}
    And then combing together, we can take the square root because the function $\sqrt{\bullet}$ is monotone increase, hence it preserves the inequality, which will give us $\Vert x\Vert_\infty \le \Vert x\Vert_2^2\le \Vert x\Vert_1$.
\section*{B.2}
    \subsection*{B.2.a}
        \textbf{Objective}: The function $\Vert x\Vert$ is a convex function. 
        \begin{align*}\tag{B.2.a.1}\label{eqn:B.2.a.1}
            \Vert \lambda x + (1 - \lambda)y \Vert 
            & \le \Vert \lambda x\Vert + \Vert (1 - \lambda) y\Vert
            \\
            & = \lambda\Vert  x \Vert +  (1 - \lambda)\Vert y\Vert
        \end{align*}
        Note, I just directly apply the Triangular inequality of the norm to get the inequality, and then becaues $\lambda\in[0, 1]$, so there is no absolute value, and notice that the resulting expression is the definition of Convexity the given function.   
    \subsection*{B.2.b}        
        \textbf{Objective}: Show that the set $\{x \in \mathbb{R}^n: \Vert x\Vert \le 1 \}$ is a convex set. Let the set be denoted as $S$ Let's take any 2 points in the set like $x \in S$,  $y \in s$, then $\Vert x\Vert \le 1$ and $\Vert y\Vert\le 1$ for any line defined by the 2 points: 
        \begin{align*}\tag{B.2.b.1}\label{eqn:B.2.b.1}
            \Vert \lambda x + (1 - \lambda)y \Vert &\le 
            \lambda \underbrace{\Vert x\Vert}_{\le \lambda} + \underbrace{(1 - \lambda)\Vert y\Vert}_{\le 1 - \lambda}
            \\
            \implies
            \Vert \lambda x + (1 - \lambda)y \Vert &\le 1
            \\
            \implies \lambda x + (1 - \lambda)y &\in S
        \end{align*}
        The first by the inequality of norm, and the second is by the definition of the fact that $x,y \in S$, and the third is by the definition of the set. 

\end{document}
