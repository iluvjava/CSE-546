\documentclass[]{article}
\usepackage{amsmath}\usepackage{amsfonts}
\usepackage[margin=1in,footskip=0.25in]{geometry}
\usepackage{mathtools}
\usepackage{hyperref}
\hypersetup{
    colorlinks=true,
    linkcolor=blue,
    filecolor=magenta,
    urlcolor=cyan,
}
\usepackage[final]{graphicx}
\usepackage{listings}
\usepackage{courier}
\lstset{basicstyle=\footnotesize\ttfamily,breaklines=true}

% \usepackage{wrapfig}
\graphicspath{{.}}

\begin{document}
\begin{center}
    Name: Hongda Li \quad Class 546 SPRING 2021\quad HW4A
\end{center}

\section*{A1: Conceptual Questions}
    \subsection*{A1.a}
        True. This is true because SVD is looking for a orthogonal matrix (PCA uses SVD) $U$, such that $U\Sigma V^T$ minimizes that reconstruction error. In this case the rank of matrix $U$ can have the same rank as the subspace span by the columns of the data matrix $X$ giving us zero reconstruction errors. 
    \subsection*{A1.b}
        True Because: 
        \begin{align*}\tag{A1.b.1}\label{eqn:A1.b.1}
            X^TX &= (USV^T)^T(USV)
            \\
            &= VS^TU^TUSV
            \\
            &= VS^TSV
        \end{align*}
        Take notice that $S^TS$ is diagonal and $V$ is orthogonal, and $X^TX$ is Symmetric. By properties of Hermitian Adjoint Matrices, it has orthogonal Eigen Decomposition with unique real eigenvectors. And $VS^TSV$ matches it, therefore $V$ is the eigen vectors of matrix $X^TX$. 
    \subsection*{A1.c}
        False. The objective should not be choosing $k$ to minimize the Loss because if $k = n$ is always the global minimum in that case and it doesn't provide any useful interpretations on the data. 
    \subsection*{A1.d}
        False. Singular values Decomposition has $U, V$ that are the eigenvectors for $XX^T$ and $X^TX$, eigen values decomposition is not unque because you can multiply eigenvector by negative one (or even worse by the complex unit $exp(i\theta)$) to get another normalized eigen vector that still works. In the case of SVD remember to flip the $u, v$ vector corssponds to the same singular value together to get different decomposition for the same matrix. 
    \subsection*{A1.e}
        False when the matrix is degenerate. In this case a eigenvalue can have a geometric multiplicity higher than it's algebaric multiplicity, then the rank of the matrix will be more than the number of eigenvalues it has. 
    \subsection*{A1.f}
        True, becaues Autoencoders with non-linear activation function can incooperate non-linear representation of data in the lower dimension. 
        
    
\section*{A2: Basics of SVD and Subgradients}
    \subsection*{A2.a}
        \subsubsection*{A2.a.(a)}
        \subsubsection*{A2.a.(b)}
    \subsection*{A2.b}
        \subsubsection*{A2.b.(a)}
        \subsubsection*{A2.b.(b)}
    \subsection*{A2.c}
\section*{A3: PCA}
    \subsection*{A3.a}
    \subsection*{A3.b}
    \subsection*{A3.c}
    \subsection*{A3.d}
    \subsection*{A3.e}

\section*{Unsupervised Learning with Autoencoders}
    
\section*{K-Mean Clustering}
    
\section*{ML in the Real World}


\end{document}
